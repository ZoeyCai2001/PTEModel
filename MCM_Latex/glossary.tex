\begin{itemize}
	\item {\bf Functional threshold power (FTP) } Maximum mean power that a cyclist can sustainably produce for the period of one hour which can be expressed in terms of watts per kilo.
	\item {\bf Power Profile Chart} It shows how does rider's maximum power output compare to the riders of various fitness and experience level, ranging from novice to the world-class, pro road cyclists.The chart includes 4 values divided by weight reflecting one's cycling talent: sprint abilities (5s), anaerobic capacity described by 1 minute maximum power, 5 minutes to tell about maximum oxygen uptake capability and 20 min to describe FTP. 
	\item{\bf Power Curve} Optimal average power output from 1 second to maximum riding time measured in watts (W) or watts per kilogram (W/kg).

	\item {\bf W} Work capacity one can expend above the FTP before reaching complete exhaustion. 
	%In our model, we assume that CP equals FTP.
	\item{\bf Time Trialist(TT)} A road bicycle racer who can maintain high speeds for long periods of time, to maximize performance during individual or team time trials. 
	\item{\bf Sprinter} A road bicycle racer or track racer who can finish a race very explosively by accelerating quickly to a high speed.
	\item{\bf Hors catégorie (HC)} It is  the most difficult type of climb in a race. The average HC climb is 15-20 kilometers long and has a grade of 9\% while the easiest category of a climb is called Category 4 with elevation gain of less than 300 meters.
\end{itemize}