\subsection{Strengths}

\begin{itemize}
	\item {\bf } Four optimization methods are applied to find optimal power distribution during cycling in the second question and they converge to same results, which ensures robustness of our model.

	\item {\bf } In the application, we compare results from virtual time trialists with Olympics racers and real-life riders which verifies the reliability and rationality of model in the second and third question.

	\item {\bf } To characterize the effect of wind direction in the model, we propose and prove a theorem based on mathematical principles, which lays the scientific foundation of sensitivity analysis in the third question.

	\item {\bf } PTE Model can be extended naturally to a variety of cycling competitions, both individual and team, which reflects broad applicability of Our model.
\end{itemize}

\subsection{Weaknesses}

\begin{itemize}
	\item {\bf } The analysis of rider's power profile and tactics based on our model can be more accurate if we have sufficient complete data about different kinds of riders' power output.
\end{itemize}

\begin{itemize}
	\item {\bf } Our model can be easily applied to tracks within half an hour to an hour of riding. However, for courses like Tour de France, our model has not been tested enough.
\end{itemize}