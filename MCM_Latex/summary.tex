\par Despite differences between types of races, minimizing the time to cover a certain course has always been the core of cycling. In our paper, Power-Time-Energy (PTE) Model is established, including three sub-models: Power Curve Fitting, Power-Time-Energy (PTE) Equation and Optimizer. PTE model analyzes the relationship between various affecting factors during cycling and rider's optimal power output.

\par Firstly, power profiles of cyclists from world-class to fair level are collected, defined as the maximum duration a rider can hold on at different exercise intensity. Based on power profiles, different cyclists' power curves are well fitted, including time trialists and sprinters of different genders. Values of adjusted $R^2$ up to 0.98 indicate the goodness of the fit.

\par Then, PTE Equation is constructed to bridge the power output and energy required in competition process based on the power profile and rider's position on the track(length and elevation gain). We next use Optimizer to produce strategy which is separated into three stages: FTP-Recovery-Sprint. To verify the rationality, we make simulation: two world-class virtual time trialists are created to participate in Olympics ITT, UCI WCTT and one self-designed race, adopting tactics provided by PTE Model. It turns out that our virtual riders rank high in these games after comparing with the real result data.

\par Next, we conduct sensitivity analysis with respect to weather conditions (wind direction and speed) and get positive feedback. Under the framework of PTE model, we put forward and prove Wind Direction Theorem. It indicates that constant direction has no effect on the total energy cost of a closed track, and in non-closed cases, the effect is only related to the location of start and end points. Moreover, optimal tactics will change appropriately according to different wind speed in our simulation.

\par Besides, the model simulates different power output at each stage and provides optimal tactics separately. The result indicates that change in total time is minor as the output power deviates to some degree. Considering the difficulty for riders to execute strategies accurately, this property ensures that the cyclist can still achieve satisfactory results under the condition of deviation.

\par Finally, our model has strong extensibility: it can also be applied to team time trial consisting of 6 riders. The whole process is divided into three stages, with 6, 5 and 4 riders uniting the team respectively. In our basic strategy, riders take turns to block the wind to keep high constant speed and the four most energetic riders will reach the end together. Similar to PTE Model, we set up corresponding equations and constraints for each stage, together with an optimization model that customizes the best tactic for a team.

 