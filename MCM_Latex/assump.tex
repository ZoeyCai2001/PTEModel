To simplify the problem and make it convenient for us to simulate real-life conditions, we make the following basic assumptions, each of which is properly justified.

\begin{itemize}
	\item {\bf The rider has a limit on total energy}.  Considering real-life conditions, it is reasonable and necessary to assume that a rider has a limited total energy.
	
	\item {\bf The rider's total energy and power curve are invariable}. To simplify the model, we assume that the rider's total energy is unchangable regardless of weather, track and other things.
	
	\item {\bf The energy needed to finish a competition is proportionate to the athlete's weight}. To simplify the model, we consider that the energy needed to finish the same course for different athletes of the same weight is unchangeable and the needed power is only associated with weight.
	
	\item{\bf The competition record of an athlete is divided into several uniform motions.} Though the speed of the rider is real-time changing, it is reasonale to simulate the process by dividing the whole time into several parts and calculate the average speed separately. 
	
	\item{\bf Wind direction and speed remain the same during a race.} In our analysis, race times usually last around an hour, so the changes in wind conditions tend to be small.
	
\end{itemize}
\newpage