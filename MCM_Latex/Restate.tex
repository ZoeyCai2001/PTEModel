\par Build a mathematical model that describes the relationship between a cyclist's position and the energy released, which can be applied to different types of riders. The model must take into account the constraints on the total energy released by the cyclist, and the limits of both past overloading and the power curve.
\begin{itemize}
	\item {\bf The first problem requires to describe power profiles of two types of cyclists}, one of which is a time trial specialist, and consider different genders.
	\item {\bf The second problem talks about the application of the model in various time trial races}, including 2021 Olympic Time Trial course, 2021 UCI World Championship time trial course and a self-designed course with at least four sharp turns and a nontrivial road grade.
	\item {\bf The third problem argues the potential impact of different environmental conditions.} It demands us to test the sensitivity of the model.
	\item {\bf The fourth problem focuses on cyclists' imperfectly execution on power distribution.} It asks to provide a detailed plan for riders.
	\item {\bf The fifth problem discusses the extension of the model.} The aim is to produce the best distribution of energy for teams of six in a team time trival.
\end{itemize}
